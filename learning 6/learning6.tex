\documentclass{ctexart}



% 正文区
\begin{document}
    
    % 在latex环境中用tabular环境来生成表格,相当于是内置区域
    \begin{tabular}{l c c c r} % 后面 l 代表向左对齐,c代表居中,r代表向右对齐
        name & age & grade & height & weight \\  % 每一列之间用&符号区分,相当于是表头,最后\\用于结束这一行继续下一行编写
        zowie & 19 & 100 & 175 & 55 \\
        alice & 19 & 100 & 160 & 50 \\
        fione & 25 & 100 & 160 & 50 \\


    \end{tabular}






    \begin{tabular}{| l || c | c | c | r |} % 如果要产生竖线则在之间加入|,基本上收尾都加|,确保表格被包裹
        \hline % \hline命令用于产生表格横线
        name & age & grade & height & weight \\  
        \hline \hline %可添加一条横线
        zowie & 19 & 100 & 175 & 55 \\
        \hline
        alice & 19 & 100 & 160 & 50 \\
        \hline
        fione & 25 & 100 & 160 & 50 \\
        \hline
    \end{tabular}


    \begin{tabular}{| l || c | c | c | r |p{2cm}} % p{xcm}用于规定本列表格的宽度,当内容超过宽度时自动填充,能自动换行
        \hline % \hline命令用于产生表格横线
        name & age & grade & height & weight \\  
        \hline \hline %可添加一条横线
        zowie & 19 & 100 & 175 & 55 \\
        \hline
        alice & 19 & 100 & 160 & 50 \\
        \hline
        fione & 25 & 100 & 160 & 50 \\
        \hline
    \end{tabular}



\end{document}