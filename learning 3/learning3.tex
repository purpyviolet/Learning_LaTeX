% 导言区
\documentclass{article}% ctexbook, ctexrep

\usepackage{ctex}

%正文区(文稿区)
\begin{document}
    \section{引言}
    介质的旋光性质反映了光与物质相互作用过程的宏观现象,由此可获得物质分子结构的重要资料。
    平面偏振光通过处于磁场中的某些物质时,振动面会发生旋转,这种现象称为法拉第磁光效应。物质的这种性质称为磁致旋光性,它表明光现象与磁现象之间有联系。旋光仪是测定物质旋光度的仪器。
    通过对样品旋光度的测定,可以分析确定物质的浓度、含量及纯度等。旋光仪广泛用于医药、食品、有机化工等各个领域,如医学上抗菌素、维生素、葡萄糖等药物分析,食品生产中食糖、味精、酱油等生产过程的控制及成品检查,等等。
    
    介质的旋光性质反映了光与物质相互作用过程的宏观现象,由此可获得物质分子结构的重要资料。
    平面偏振光通过处于磁场中的某些物质时,振动面会发生旋转,这种现象称为法拉第磁光效应。物质的这种性质称为磁致旋光性,它表明光现象与磁现象之间有联系。旋光仪是测定物质旋光度的仪器。
    \\通过对样品旋光度的测定,可以分析确定物质的浓度、含量及纯度等。旋光仪广泛用于医药、食品、有机化工等各个领域,如医学上抗菌素、维生素、葡萄糖等药物分析,食品生产中食糖、味精、酱油等生产过程的控制及成品检查,等等。
    %反斜杠用于换行,但是不是换为新的一段

    % 使用\par 可以用于换成新的段落

    \section{实验方法}
    \section{实验结果}

    \subsection{数据}
    \subsection{图表}
    \subsection{结果分析}
    \subsubsection{实验条件}
    \subsubsection{实验过程}

    \section{结论}
    \section{致谢}

\end{document}