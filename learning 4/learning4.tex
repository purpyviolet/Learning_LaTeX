% 导言区
\documentclass{article}

\usepackage{ctex}%中文处理宏包
% \usepackage{xltxtra}

% 正文区(文稿区)
\begin{document}
    \section{空白符号}
    Hello Here is       purpy violet!
    %这里注意不管你中间添加了多少空格,都是正常显示一个空格

    你好这里是purpy violet正在说话!
    %中文与其他字符混合的话,具体看编译器,可能会自动添加空格,可能不会

    你好这里是      坡儿皮薇尔莉特!
    %中文字符之间不管添加多少空格都是默认没有空格

    % 如果要实现空格,可以用以下的方法

    % 1. 1em(相当于字体中M的宽度)因为M是所有字母中最宽的那一个,所以便用M作为基准
    a\quad b

    % 2. 2em
    a \qquad b

    % 3. 约为1/6个em
    a\,b a\thinspace b

    % 0.5个em
    a\enspace b

    % 直接一个正常空格的宽度
    a\ b

    % 硬空格
    a~b

    % 详细规定空格距离(用\kern \hskip \hspace等指令)
    a\kern 1pc b % 1pc=12pt=4.218mm

    a\kern -1em b % 则把后面的所有东西往前平移一个em距离

    a\hskip 1em b

    a\hspace{35pt} b

    %占位宽度
    a\hphantom{xyz}b %即中间有个xyz宽度的占位空格

    % 弹性宽度(占据整个横向空间)
    a\hfill b



    \section{\LaTeX 控制符}
    \# 
    \$ 
    \% 
    \& 

    % 打出{}
    \{\} 

    % 打出~
    \~{} 

    % 打出_
    \_{} 

    % 打出^
    \^{} 

    % 打出\
    \textbackslash



    \section{排版符号}
    % section符号
    \S

    % idk what is this
    \P

    % sword
    \dag

    % fence
    \ddag

    % copyright符号
    \copyright

    % 英镑符号
    \pounds




    \section{\TeX 标志符号}
    \TeX {} \LaTeX{} \LaTeXe{}
    % 如果是\XeLaTeX的话就需要额外导入{xltxtra}宏包,包括XeLaTeX的一些标志符号
    \section{引号}
    % 这里直接按引号的那个键并不能区分左右引号,所以这里改用~`'键位
    ` ' ``  ''

    % 直接用‘“键位打出来的符号
    ' ' " "

    % 如果是中文的字符引号的话就不用在意了
    “” “” ‘’
    \section{连字符}
    % 长度为1
    -

    % 长度为2
    --

    % 长度为3
    ---
    \section{非英文字符}
    %
    \oe ~ \OE ~ \ae ~ \AE ~ \aa ~ \AA ~ \o ~ \O ~ \l ~ \L ~ \ss ~ \SS ~ !` ?`

    \section{重音符号}
    \`o % 重音符号:重音符号(\`)放在字母(o)之前,表示 Grave(重音符)。
    \'o % 重音符号:重音符号(\')放在字母(o)之前,表示 Acute(锐音符)。
    \^o % 重音符号:重音符号(\^)放在字母(o)之前,表示 Circumflex(抑扬符)。
    \"o % 重音符号:重音符号(\")放在字母(o)之前,表示 Umlaut(分音符)。
    \~o % 重音符号:重音符号(\~)放在字母(o)之前,表示 Tilde(波浪符)。
    \=o % 重音符号:重音符号(\=)放在字母(o)之前,表示 Macron(长音符)。
    \.o % 重音符号:重音符号(\.)放在字母(o)之前,表示 Dot(点音符)。
    \u{o} % 重音符号:u 上加 breve(短音符)。
    \v{o} % 重音符号:v 上加 caron( háček)。
    \H{o} % 重音符号:H 上加 double acute(双重音符)。
    \r{o} % 重音符号:r 上加 ring(圆圈符)。
    \c{o} % 重音符号:c 下加 cedilla(锐音符)。
    \d{o} % 重音符号:d 下加 dot-under(下圆点符)。
    \b{o} % 重音符号:b 下加 bar-under(下横线符)。
    \t{o} % 重音符号:t 上加 tie-over(横线符)。
    \H{o} % 重音符号:H 上加 Hungarian umlaut(匈牙利分音符)。


\end{document}