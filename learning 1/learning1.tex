% 导言区
\documentclass{article} %或者可以是book, ctexart, report, letter等等类
% “%”为注释符号

% 用\title{title}可以为文章加上标题
\title{\heiti How to learn LaTeX part 1}% 当然如果你要修改字体的话就在前面加上"\字体名称" 就行,后面接一个空格分隔开来
% 用\author{name}可以为文章加上作者
\author{purpy violet}
% 用\date{date}可以为文章加上日期
% 其中\date{\today}则代表今天的日期
\date{\today}

% 关于在文章中使用中文,我们需要引用一个宏包,即为\usepackage指令,宏包名字为ctex
\usepackage{ctex} %当然如果你想一步到位你可以直接使用ctexart文章类就行,就不用写这个了    
% 同时需要注意编码格式为UTF-8,排版为XeLaTeX

% 这里可以引用一些新的指令,比如说后面用到的弧度的组合公式表达,这边提前定义一下后面就省了很多事了,语法为\newcommand\新表达名称{旧的组合表达}
\newcommand\degree{^\circ}

% 当然要注意了,你在导言区定义了标题,作者和日期
% 可是它们并没有被print出来,你需要在下面正文区才能调用显示它们




% 正文区
\begin{document} %{}内为环境名称,一般设置为document。一个LaTeX文件中只能有一个document环境
		% 在这里用\maketitle指令打印出标题日期等信息
		% 注意在letter类中没有这个指令哦
		\maketitle
		%注意没有\makedate或者\makeauthor等指令,你只要maketitle就可以把日期作者什么的一起打印出来了

		Hello World!

		%关于换行操作的话就直接在这边空一行就行,当然不可以是注释行,要纯空白的
		%两个$之间塞数学公式就不用多说了,四个$就是单独占用一行的数学公式
		Let $f(x)$ be defined by the formula 
		$$f(x) = 3x^2 + x - 1$$ which is a polynomial of %这也代表着在代码区换行并不会导致打印出来的文章会换行,of 后面接着degree 2一起连着输出。
		degree 2. 

% 以下为中文部分的显示

接受一个人就接受她的全部。美好的,惨痛的,幸福的,痛苦的······一切的一切,都要全盘接受……就和做卷子一样,不能只做想做的题。只要你拿起一张卷子,就要从头到尾把它做完。
%我知道你每次看到代码区没有自动缩进会很难受,但是这并不影响lol


勾股定理可以用符号语言表述为:设直角三角形$ABC$,其中$\angle C =90^\circ$,则有:%\circ代表数学中的“°”




\begin{equation} %特别注意这里的公式写法,我们引用了\begin{equation}和\end{equation},这样子生成的公式会自动生成一个公式编号
AB^2 = BC^2 + AC^2 
\end{equation}

这里关于公式的引用也涉及到一些像是函数重载之类的操作,比如像是上面的$\angle C =90^\circ$,其中$^\circ$代表上角标的°,但是打起来比较麻烦,所以说我们可以定义一个新指令来代替$^\circ$,在导言区加入新指令的声明,可以方便很多。

\end{document}
