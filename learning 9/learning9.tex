\documentclass{ctexart}
\usepackage{amsmath}
\begin{document}
    $
    \begin{matrix}% 注意这里需要加$,其实矩阵本质上也是公式,不能漏了$
        1 & 0 \\
        0 & 1
    \end{matrix} % matrix环境本身不包括任何括号,只有数字
    $

    $
    \begin{pmatrix}% pmatrix用于在矩阵两端加上小括号
        2 & i \\
        3 & 5
    \end{pmatrix}
    $

    $
    \begin{bmatrix}% bmatrix用于在矩阵两端加上中括号 
        8 & i \\
        3 & 678
    \end{bmatrix} % 最常用!!!
    $

    $
    \begin{Bmatrix}% Bmatrix用于在矩阵两端加上大括号
        8 & i & t\\
        33 & 678 & y
    \end{Bmatrix}
    $

    $
    \begin{vmatrix}% vmatrix用于在矩阵两端加上绝对值,在矩阵中也就是判别式
        8 & -5i \\
        3 & 68
    \end{vmatrix}
    $

    $
    \begin{Vmatrix}% Vmatrix用于在矩阵两端加上‖
        2 & -5i \\
        -i & 64
    \end{Vmatrix}
    $

    % 使用上下角标以及参数接收
    $
    B = \begin{bmatrix}
        2 & 45 & b_{23} \\
        34 & a_{4} & b_{1} \\
        23 & 90 & 3
    \end{bmatrix}
    $

    % 常用省略号:\dots、\vdots、\ddots
    % \dots即为正常横向省略号,\vdots即为vertical竖直省略号,\ddots即为斜着的省略号
    % 这里与\cdot区分,没有s的只是单个点,\cdots与\dots在使用上没有什么不一样
    $
    A = \begin{pmatrix}
        a_{1} & \dots & a_{1n} \\
        & \ddots & \vdots \\
        0 & & a_{nn}
    \end{pmatrix}_{n \times n} % 这里可以把矩阵本身也看成是一个式子,进行操作
    $

    % 分块矩阵
    $
    \begin{pmatrix}
        \begin{matrix} 
            1 & 0 \\
            0 & 1
        \end{matrix} & \text{\Large 0}\\
        \text{\Large 0} & 
        \begin{matrix}
            1 & 0\\
            0 & -1
        \end{matrix}
    \end{pmatrix}
    $

    % 三角矩阵
    $
    \begin{pmatrix}
        a_{11} & a_{12} & \cdots & a_{1n} \\
        & a_{22} & \cdots & a_{2n} \\
        &      & \ddots & \vdots \\
        \multicolumn{2}{c}{\raisebox{1.3ex}[0pt]{\Huge 0}} % 2为两列,c为居中
        &      & a_{nn}
    \end{pmatrix}
    $

    % 跨列用省略号 \hdotsfor{columns}
    $
    \begin{bmatrix}
        2 & 45 & b_{23} \\
        34 & a_{4} & b_{1} \\
        \hdotsfor{3} \\
        23 & 90 & 3
    \end{bmatrix}
    $

    % 行内小矩阵用smallmatrix环境来实现,但是需要手动添加左右括号
    复数$z = (x,y)$也可以用矩阵
    \begin{math}
        \left (
            \begin{smallmatrix}
                x & y \\
                y & x
            \end{smallmatrix}
        \right )
    \end{math}
    来表示。

    % 用array环境来排版(与tabular相似)
    $
    \begin{array}{r|r}
        \frac12 & 0 \\ % 这里直接跟了12两个字符,系统默认为{1}{2}
        \hline 
        0 & -\frac abc \\
    \end{array}
    $
\end{document}

